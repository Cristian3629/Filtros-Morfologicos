\documentclass[10pt,a4paper]{article}
\usepackage[latin1]{inputenc}
\usepackage[spanish]{babel}
\usepackage{amsmath}
\usepackage{amsfonts}
\usepackage{amssymb}
\usepackage{graphicx}
\usepackage{vmargin}
\setpapersize{A4}
\setmargins
{2.5cm}       % margen izquierdo
{1.5cm} % margen superior
{16.5cm}% anchura del texto
{23.42cm} % altura del texto
{10pt} % altura de los encabezados
{1cm} % espacio entre el texto y los encabezados
{0pt} % altura del pie de p�gina
{2cm}% espacio entre el texto y el pie de p�gina

\begin{document}
	
	\begin{titlepage}
		\centering
		{\scshape\LARGE Universidad de Buenos Aires \par}
		\vspace{1cm}
		{\scshape\Large Taller de Programaci�n (75.42)\par}
		\vspace{1.5cm}
		{\huge\bfseries Filtro Morfol�gicos \par}
		\vspace{2cm}
		{\Large\itshape Cristian Gonz�lez\par}
		\vspace{2cm}
		{\Large\itshape 94719 \par}
	\end{titlepage}
	
	
	
	
	
	
	%�ndice.
	\tableofcontents
	\newpage
\section{Clases}
\subsection{Interpreter}
La responsabilidad de esta clases es dado en un string que dentro tiene el formato de una matriz, poder interpretarlo y crear una matrix con el formato indicado.

\subsection{Filter}
Es una clase abstracta que se encarga de dado una matrix imagen y una matrix patr�n encargarse de constatar una matrix con la otra, tiene el compartimiento general de los filtros del trabajo pr�ctico.

\subsection{Erosion}
Modela el comportamiento del Filtro Erosion que busca la coincidencias totales para emitir en la matrix destino.
\subsection{Dilatation}
Se encarga de modelar el comportamiento del Filtro Dilataci�n y de tal manera buscar coincidencia parciales para luego emitir en la matrix destino.

\subsection{Matrix}

Es el repositorio de la "imagenes" tanto el de origen, patr�n y destino, b�sicamente es una matrix que contiene strings.

\subsection{Position}

Simplemente modela las coordenada de una matrix (fil,col) y realiza operaciones b�sica como suma de posiciones como tambi�n obtener la posici�n relativa.

\subsection{Thread}
Se encarga de de modelar el comporamiento de un thread, para crearlo se tiene que proveer los datos que va a necesitar ese thread como la funci�n/m�todo a ejecutar.


\subsection{FiltrosMorfologicos}
Clase punto de entraga al sistema, se encarga de procesar lo que se ingresa por STDIN, como controlar la cantidad de hilos y aplicar de forma secuencial los filtros que correspondan.

\newpage
\section{Diagramas}
Ac� se muestra como se relaciona el filtro con las otras clases, el Filter utiliza a la Matrix para poder conocer sus dimensiones, poder crear Matrix vac�a, como poder obtener elemento de cierta Position.\\
Tanto como Dilatation como Erosion definen el m�todo 
\texttt{checkCoincidence()} ya que cada filtro busca distintas coincidencias, esto es conocido como el patr�n \underline{Template Method}.\\


\begin{figure}[h!]
\centering
\includegraphics[width=1.0\linewidth]{Filtro}
\label{fig:Filtro}
\end{figure}

\newpage
La clase principal interact�a con 3 clases:Filter,Interpreter y Matrix. Con Interpreter se encarga de identificar qu� clase de Filter a aplicar y crea un instancia de la misma, para luego aplicar un filtro a la Matrix ingresada por STDIN junto con el patr�n devuelto por Interpreter. A Matrix lo utiliza para poder crear una instancia que se copiar a partir de otra y poder setear todos los valores de una Matrix a partir de una instancia que se le pasa.

\begin{figure}[h]
\centering
\includegraphics[width=1.0\linewidth]{FiltroFormologicoPrincipal}
\label{fig:FiltroFormologicoPrincipal}
\end{figure}


\section{Dificultades}
\begin{enumerate}
	\item Al crear un Matrix utiliza \texttt{malloc()} para poder reservar memoria, eso generaba \texttt{jump conditional} que si bien es memoria reservada, puede tener basura.
	\item Al utilizar el patr�n Template Method, la idea inicial era poder simplemente conocer un Filter y aplicar el filtro, pero para eso tuve que recurrir al \texttt{new}, ya que s� o s� necesito un puntero para que llame al m�todo definido en la clase base.
	\end{enumerate}
	
\section{Cambios}
\begin{itemize}
	\item En Matrix hay un m�todo set, este en realidad deber�a ser una sobrecarga del \texttt{operator \textbf{=}} siendo un \texttt{asignaci�n por movimiento}.
	\item El m�todo \texttt{createMatrix()} deber�a ser un \texttt{constructor por movimiento}.
	\item El m�todo \texttt{aplicateFilter()} deber�a ser un \texttt{constructor por movimiento}.
	\item Para los \texttt{Threads} la idea es crear un container que tenga los datos necesarios para \texttt{Threads}, y que sea accesible por \texttt{Singleton}.
\end{itemize}

\section{Links}
Link del repositorio:https://github.com/Cristian3629/Filtros-Morfologicos

\end{document}